% to w założeniu ma się skompilować Heveą, zwykły pdftex nie przejdzie
\documentclass[10pt,a4paper,utf8]{article}
\usepackage{polski}
\usepackage[utf8]{inputenc}
\usepackage{hyperref}

\renewcommand{\contentsname}{Spis treści}

\begin{document}

\title{Testowe coś}
\date{1 IV 2020r}
\author{Ziom}
\maketitle
\tableofcontents
\section{Wstęp}\label{sec:wstep}
No elo ziomeczki\footnote{i resztę też witam}.
Jak tam u was? \\
\subsection{pod}
Kompilowane  \href{http://hevea.inria.fr/}{Heveą}. wynik: \href{../../index.html}{qwerty} \url{../../index.html}
\subsubsection{xD}
łąœðπ©ęþŋ→ß’–©œ→
\section{Zakończenie}\label{sec:zakonczenie}
No to na
\( \frac{a}{b} = \Pi_{(x : A)}{B(x)}^2 \)
razie!
\[
\frac{a}{b} = \Pi_{(x : A)}{B(x)}^2
\]
papa, początek był tu: \ref{sec:wstep}. Albo inaczej: \hyperref[sec:wstep]{inaczej \ref{sec:wstep} }.
\end{document}
